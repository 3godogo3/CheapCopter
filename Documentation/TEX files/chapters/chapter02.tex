\section{Design}
    Tvorba PCB začala schématem zapojení všech komponent. K tomuto úkolu jsem využil software KiCad, který je velmi intuitivní i pro úplné začátečníky. Dále je potřeba navrhnout samotné PCB. Pro náš projekt bude dostatečné nejlevnější řešení PCB a to pevné dvouvrstvé bez zlatých pinů. Po doporučení vedoucího práce byl přidán GND do celé plochy tištěného spoje, kde se nevyskytují jiné propoje, pro jednoduché propojení všechs komponent. 

\section{Objednání}
    Pro samotné vytvoření tohoto tištěného spoje jsem si vybral stránku JLCPCB.COM, která umožňuje za velmi nízkou cenu(cca 100 Kč za pět kusů PCB), nechat si vyrobit jakékoliv PCB. Jediná nevýhoda je lokace společnosti. Sídlí v Číně, a proto doprava trvala tři týdny. Byla možnost si připlatit za doručení do několika dnů, ale částka by se vyšplhala i na 1000 Kč za samotnou dopravu, což je skoro vyšší částka než hodnota samotného dronu.
