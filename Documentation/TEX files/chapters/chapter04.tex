V této kapitole si popíšeme jak dron ovládat a zdali opravdu funguje.

\section{Připojení se}
    Po zapojení baterie se dron automaticky zapne. Několik vteřin na to se vytvoří Wi-Fi access point, který se základně jmenuje "Drone" a lze se k němu připojit za pomocí hesla "password". Jak jméno, tak heslo pro AP se dá změnit v kódu přepsáním proměnných *ssid a *password. Pro možnost nahrání upraveného kódu na ESP, si otevřete webový interface na adrese 192.168.4.1/update. Zde můžete nahrát vámi upravenou binárku. Pro možnost ovládání drona, se musíte připojit přes protokol Telnet v příkazové řádce či jiném programu, jako například PuTTY, podporující Telnet. V našem případě budem používat samotnou příkazovou řádku. Pokud nemáte už Telnet nainstalovaný, postupujte dle návodu na stránce \parencite[Telnet download tutorial]{telnet}. Za pomocí následujícího příkladu započnete vzdálenou komunikaci s dronem.
    
    \begin{lstlisting}

    $ telnet 192.168.4.1 23
    # Trying 192.168.4.1...
    # Connected to 192.168.4.1.
    # Escape character is '^]'.

    \end{lstlisting}

\newpage

\section{Ovládání}
    V tento moment můžete používat všechny nastavené příkazy.\\ Mezi tyto příkazy patří:\par "w" pro let dopředu,\par "s" pro let dozadu,\par "d" pro let do prava,\par "a" pro let do leva,\par "r" pro let nahoru a\par "f" pro let dolu.

    \begin{lstlisting}

    $ w
    # Received command forwards
    $ a
    # Recieved command left
    $ s
    # Recieved command backwards
    $ r
    # Recieved command up
       
    \end{lstlisting}

\newpage

\section{Skutečné použití}
     I přes veškerou snahu se dron není schopen sám zvednout. Je tomu tak z důvodu příliš vysoké hmotnosti drona, ze kterého motory nemají dostatečný tah pro vzlet. Více než polovina hmotnosti je tvořena samotnou baterií. Dále by bylo dobré zvážit použití jiných vrtulí. Všechny ostatní funkce fungují. Jediné co by na dronu s nižší hmotností bylo potřeba doladit jsou konstanty u vyvažování pomocí gyroskopu a základní rychlost motorů, kterou se otáčejí bez zadaných příkazů.