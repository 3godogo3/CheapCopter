\section{Rám}
    Pro správnou funkčnost drona je zapotřebí správný poměr pevnosti a hmotnosti těla. Tohoto jsem dosáhl odhadem, kde první verze měla plné dno, čímž byla konstrukce rámu značně pevnější, ale o několik desítek gramů těžžší. V druhé verzi jsem ztenčil všechny stěny, ubral materiál dna a přidal nožicky pod jednotlivé motory pro lepší ochranu drátů motorů. Tímto se sice snížila pevnost, ale díky robustnosti samotného PCB to není znát, a zároveň se snížila hmotnost. Jako poslední jsem se rozhodl využít materiál PLA, se kterým se jednodušeji tiskne a je levnější, protože vyšší pevnost jiných materiálů jako například ABS mi nepřipadala nutná. Veškerá tvorba modelů byla provedena v programu Autodesk Fusion 360 a Prusa Slicer.

\section{Vrtule}
    Při první verzi modelu jsem se pokusil vymodelovat a vytisknout i vrtule, ale po dlouhé snaze, se mi nedařilo vytvořit ideální tvar. Z toho důvodu jsem využil model z internetu. Po jeho vytisknutí a testování jsem se ale dostal k závěru, že s mně dostupným materiálem jsou vrtule příliš tvrdé a křehké, a proto jsem se uchýlil ke koupi 2" vrtulí. Tyto vrtule jsem vybral po konzultaci s poradcem z FYFT.CZ.